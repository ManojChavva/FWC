\documentclass[journal,12pt,twocolumn]{IEEEtran}
\usepackage{graphicx}
\usepackage{listings}
\usepackage[utf8]{inputenc}
\usepackage{caption}
\usepackage{hyperref}
\usepackage[cmex10]{amsmath}
\usepackage{array}
\usepackage{gensymb}
\usepackage{booktabs}
\usepackage{etoolbox}
\patchcmd{\section}{\centering}{}{}{}
\providecommand{\norm}[1]{\left\lVert#1\right\rVert}
\providecommand{\abs}[1]{\left\vert#1\right\vert}
\let\vec\mathbf
\newcommand{\myvec}[1]{\ensuremath{\begin{pmatrix}#1\end{pmatrix}}}
\newcommand{\mydet}[1]{\ensuremath{\begin{vmatrix}#1\end{vmatrix}}}
\providecommand{\brak}[1]{\ensuremath{\left(#1\right)}}

\title{Matrix Problems \textbf{\\Straight Lines }}
\author{Manoj Chavva} 

\begin{document}
\maketitle

Get Python code for the figure from 
\begin{lstlisting}
https://github.com/SurabhiSeetha/Fwciith2022/tree/main/Assignment%201/codes/src
\end{lstlisting}
Get LaTex code from
\begin{lstlisting}
https://github.com/SurabhiSeetha/Fwciith2022/tree/main/avr%20gcc
\end{lstlisting}

\section{Problem Statement}

\noindent The base of an equilateral triangle with side 2a lies along the y-axis such that the mid-point of the base is at the origin. Find vertices of the triangle.


\section{Solution}
\noindent Given ABC is an equilateral triangle i.e 
\begin{equation}
AB = BC = CA  
\end{equation}

\begin{figure}[h]
\includegraphics[width=1\columnwidth]{triangle.png}
\caption{Equilateral Triangle ABC}
\label{fig:triangle}
\end{figure}

\section{Construction}
B and C are the inputs.
\begin{table}[h]
\centering
\large
\begin{tabular}{|l|l|l|}
\hline
\textbf{Symbol} & \textbf{Value} & \textbf{Description} \\ \hline
B               & (0, 2)         & Vertex B             \\ \hline
C               & (0, -2)        & Vertex C             \\ \hline
A               & (x,y)          & Vertex A             \\ \hline
A1              & (x1, y1)       & Vertex A1            \\ \hline
\end{tabular}
\end{table}

\noindent Since base with 2a is lies on the y-axis with the mid-point of the base is at origin. The vertices of the two points on y-axis will be

\begin{equation}
\vec{B}=\begin{pmatrix} 
0\\
a
\end{pmatrix}, {
\vec{C}=\begin{pmatrix} 
0\\
-a
\end{pmatrix} }
\end{equation}
%$\vec{B}(0 , a)$ and $\vec{A}(x, y)$ 

%\noindent It is known that the line joining a vertex of an equilateral triangle with the midpoint of its opposite side is perpendicular. \\


%\noindent Since, the mid point lies on origin. The vertex \textbf{A} lies on the x-axis.\\

%\noindent So, The vertex will be \\


%begin{equation}
%\vec{A}=\begin{pmatrix} 
%x\\
%0
%\end{pmatrix}
%\end{equation}

\noindent The distance between the two points B and A is
%\begin{equation}
%\vec{B} = \myvec{0 \\ a}, \vec{A}=\myvec{x \\ 0}
%\end{equation}

\begin{equation}	
\vec{B}-\vec{A} = \myvec{0-x \\ a-y}
\end{equation}

\noindent Using the definition   of the norm, 
		\begin{equation}
\norm{\vec{B}-\vec{A}} =\norm{\myvec{-x \\ a-y}}
\end{equation}
\noindent Since, the side of an equilateral triangle is 2a	
\begin{equation}						
			2a=\sqrt{\myvec{-x & a-y}\myvec{-x \\ a-y}} 
\\
\end{equation}
\begin{equation}						
2a =  \sqrt{(x)^2+ (a-y)^2}
\end{equation}
\noindent Squaring on both sides

\begin{equation}						
4a^2 =  {(x)^2+ (a-y)^2}
\end{equation}
\begin{equation}
4a^2 = {x^2 + a^2 +y^2-2ay}
\end{equation}
\begin{equation}
3a^2 = {x^2+y^2-2ay}
\label{eq-1-}
\end{equation}

\noindent Similarly, The distance between the two points C and A is

\begin{equation}	
\vec{C}-\vec{A} = \myvec{0-x \\ -a-y}
\end{equation}

\noindent Using the definition   of the norm, 
		\begin{equation}
\norm{\vec{C}-\vec{A}} =\norm{\myvec{-x \\ -a-y}}
\end{equation}
\noindent Since, the side of an equilateral triangle is 2a	
\begin{equation}						
			2a=\sqrt{\myvec{-x & -a-y}\myvec{-x \\ -a-y}} 
\end{equation}
\begin{equation}						
2a =  \sqrt{(x)^2+ (a+y)^2}
\end{equation}
\noindent Squaring on both sides
\begin{equation}						
4a^2 =  {(x)^2+ (a+y)^2}
\end{equation}
\begin{equation}
4a^2 = {x^2 + a^2 +y^2+2ay}
\end{equation}
\begin{equation}
3a^2 = {x^2+y^2+2ay}
\label{eq-2-}
\end{equation}


Solving equation (\ref{eq-1-}) and (\ref{eq-2-}), we get
\begin{equation*}						
x =  \pm\sqrt{3}a 
\end{equation*}
\begin{equation}						
y=0
\end{equation}
Hence,the coordinates of the vertices of triangle are A(\sqrt{3}a,0),B(0,a) and C(0,−a) \\
  or\\
A(−\sqrt{3}a,0),B(0,a) and C(0,−a).

\end{document}
