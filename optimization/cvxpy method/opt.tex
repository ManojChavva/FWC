\documentclass[journal,12pt,twocolumn]{IEEEtran}
\usepackage{graphicx}
\usepackage{listings}
\usepackage[utf8]{inputenc}
\usepackage{caption}
\usepackage{hyperref}
\usepackage[cmex10]{amsmath}
\usepackage{array}
\usepackage{gensymb}
\usepackage{booktabs}
\usepackage{etoolbox}
\usepackage{amssymb}
\usepackage{datetime} 
\patchcmd{\section}{\centering}{}{}{}
\providecommand{\norm}[1]{\left\lVert#1\right\rVert}
\providecommand{\abs}[1]{\left\vert#1\right\vert}
\let\vec\mathbf

\makeatletter
\newcommand\xleftrightarrow[2][]{%
  \ext@arrow 9999{\longleftrightarrowfill@}{#1}{#2}}
\newcommand\longleftrightarrowfill@{%
  \arrowfill@\leftarrow\relbar\rightarrow}
\makeatother
\title{Optimizb,b,bation}
\author{Manoj Chavva} 
\date{october 2022} 
\newcommand{\myvec}[1]{\ensuremath{\begin{pmatrix}#1\end{pmatrix}}}
\newcommand{\mydet}[1]{\ensuremath{\begin{vmatrix}#1\end{vmatrix}}}
\providecommand{\brak}[1]{\ensuremath{\left(#1\right)}}
\providecommand{\lbrak}[1]{\ensuremath{\left(#1\right.}}
\providecommand{\rbrak}[1]{\ensuremath{\left.#1\right)}}
\providecommand{\sbrak}[1]{\ensuremath{{}\left[#1\right]}}

\begin{document}
\maketitle
\section{Problem Statement}

\noindent A wire of length 50 m is cut into two pieces. One piece of the wire is bent in the shape of a square and the other in the shape of a circle. What should be the length of each piece so that the combined area of the two is minimum \\

\noindent \textbf{To Find:} \\
The value of the length of each piece so that the combined area of the two is minimum from the two figures that are square and circle.

\noindent \textbf{Given:} \\
Length of the wire is 50m
\section{solution}
\begin{equation}
\text{length of the square is} \quad \vec{a} \quad \text{m} .
\end{equation} 
Then length of the other piece for the shape of the circle is \begin{equation}
(50-a) \text{m}
\end{equation} 
Perimeter of the square with side a is given by: \\
\begin{equation}
\text{Perimeter of the square = } 4a \\
\label{eq-1}
\end{equation}
%
%Now, we know from the above condition that the total length of all four sides of the square is x.\\
%
%Then, by using equation \ref{eq-1} we get:
%\begin{equation}
%x = \frac{a}{4}
%\end{equation}

Similarly, we know the formula for the circumference of the circle with radius r is given by:
\begin{equation}
\text{Circumference of a circle= }2\pi r
\label{eq-2}
\end{equation}
So, the total length is 
\begin{equation}
4x + 2\pi r = 50
\end{equation} 
The standard equation of the line in conics is given as :
\begin{align}
n^\top \vec{x} = c
\end{align}
\begin{equation}
\begin{pmatrix}1 & 2\pi\end{pmatrix}  \vec{x} = 25
\end{equation}

\begin{align}
\vec{x}  = \myvec{x\\r}
\end{align}
%Now, we know from the above condition that the total length of all circles is (50-x).
%Then, by using equation \ref{eq-2} we get:
%\begin{equation}
%r=\frac{50-x}{2\pi}
%\end{equation}
%%The standard equation of the line in conics is given as :
%\begin{align}
%n^\top \vec{x} = c
%\end{align}
%\begin{equation}
%\begin{pmatrix}1 & 2\pi\end{pmatrix}  \vec{x} = 50
%\end{equation}
%\begin{align}
%\vec{x}  = \myvec{x\\r}
%\end{align}
Now by using the formula for the area of the circle and square is:
\begin{equation}
\text{Area of square= }a^2
\end{equation}
\begin{equation}
\text{Area of the circle= }\pi r^2
\end{equation}
Now, the combined area(A) 
\begin{equation}
A=a^2+\pi r^2
\end{equation}
%\begin{equation}
%A = \frac{x^2}{16} + \frac{(50-x)^2}{4\pi}
%\end{equation}
The area of two figures is grepresented as :
\begin{align}
\vec{x}^{\top}\vec{V}\vec{x}+2\vec{u}^{\top}\vec{x}+f=0
\end{align}
\begin{equation}
\vec{V} = \begin{pmatrix}
1 & 0 \\
0 & \pi
\end{pmatrix}
\end{equation}
\begin{equation}
u^\top = \begin{pmatrix}
0 & 0
\end{pmatrix}
\end{equation}
\begin{equation}
f = 0
\end{equation}
\noindent The minimum area is 
\begin{align}
\min_{x} \vec{x}^{\top}\vec{V}\vec{x} 
\end{align}
\noindent Such that, 
\begin{equation}
\begin{pmatrix}1 & 2\pi\end{pmatrix}  \vec{x} -25 == 0
\end{equation}
Solving using cvxpy, we get
\begin{equation}
\min_{x} \vec{x}^{\top}\vec{V}\vec{x} = 87.53 \quad m^2
\end{equation}
The length of each piece is \\
Square = 4a = 28 m \\
circle = 2 $\pi$ r = 21.98 m  
%The minimum value is caluculated by using gradient descent method.
%\begin{align}
%        x_{n+1} &= x_n - \alpha \nabla f(x_n) \\
%        \implies x_{n+1} &= x_n - \alpha\brak{\frac{x}{8}-\frac{50-x}{6.28}}
%\end{align}
%where \\
%\begin{enumerate}
%\item $\alpha$ = 0.001
%\item $x_{n+1}$ is current value
%\item $x_{n}$ is previous value
%\item precession = 0.00000001
%\item maximum iterations = 100000000
%\end{enumerate}
%The minimum values obtained from the python code \vspace{5mm}\\
%The given function has minimum value at 28.011 i.e
%\begin{equation}
%\frac{200}{4+\pi}
%\end{equation}
%\begin{figure}[h]
%\includegraphics[width=1\columnwidth]{opt2.png}
%\label{fig:triangle}
%\end{figure}
%Then, the length of the circle wire is 50-x and substitute the x value in it, we get:
%\begin{equation}
%\frac{50\pi}{4+\pi}
%\end{equation}
%Hence, the length of square is $\frac{200}{4+\pi}$ m and circle wire is $\frac{50}{4+\pi}m$ to get the minimum area.
\end{document}